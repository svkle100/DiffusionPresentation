\section{Ergebnispräsentation}

\begin{frame}
  \frametitle{Grafiken}
  Grafiken werden wie gewohnt eingebunden (nur ohne \texttt{figure}-Umgebung):
  \begin{center}
	    \includegraphics[width=0.5\textwidth]{example-image-a}
  \end{center}
\end{frame}

\begin{frame}{Grafiken mit Beschriftung}
Oder alternativ mit \texttt{figure}-Umgebung:
\begin{figure}
\includegraphics[width=0.4\textwidth]{example-image-b}
\caption{Ein Beispielbild}
\end{figure}
\end{frame}

\begin{frame}{Quellenangaben}
Falls es relevant ist, kann man eine Literatur-Quelle~\footfullcite{Krauthoff2017a} auf den Folien angeben.

Dabei ist es praktisch, wenn die komplette Zitationsangaben auf der Folie selbst steht, ansonsten muss das Publikum bis zum Ende warten, um die Quellennummer aufzulösen.
\end{frame}

\begin{frame}[fragile]{Code} % verbatim-Umgebungen sind fragil!
    Code ist bei den meisten Arbeiten nicht notwendig. Eine 0815-Implementierung einer Funktionalität interessiert nicht. Für schlaue Algorithmen sind Flussdiagramme oder Beispiele oft sinnvoller als konkreter Code.
    
    Wenn der Code selbst aber von Interesse ist, z.\,B. weil eine bestimmte Programmiertechnik oder eine Programmiersprache vorgestellt werden, sind Codeblöcke sinnvoll.
    
\begin{minted}{python}
def foo(x: int) -> int:
    return x ** x
\end{minted}

\inputminted[linenos=true]{java}{example.java}

\begin{lstlisting}[language=python]
def foo(x: int) -> int:
    return x ** x
\end{lstlisting}
\end{frame}

\begin{frame}{Notizen}
  Notizen können mit \texttt{note} hinzugefügt werden. Diese werden gerendert, wenn in \texttt{master.tex} die Zeile mit \texttt{show notes on second screen} aktiviert wird.
  
  Zum Präsentieren mit Notizen eigenen sich spezielle Programme wie PDF Presenter Console (pdfpc).
  
  \note{Notizen}
\end{frame}
